\section{Implementasi}

\subsection{Lingkungan Pendukung}
\begin{itemize}
	\item \textit{Framework}
\\Sistem informasi ini adalah sistem informasi berbasis web. Sistem informasi ini dibangun dengan \textit{framework} Laravel 4.2\footnote{http://laravel.com/docs/4.2/quick}. 
	\item Basis Data
\\Basis data yang digunakan selama proses pmbangunan sistem ini adalah MySql denagan dilengkapi antarmuka PhpMyAdmin.  
	\item \textit{Web Server} 
\\Contoh: Apache, NGINX
\end{itemize}

\subsection{Petunjuk Instalasi}
\begin{enumerate}
	\item Buka MySql (boleh lewat antarmuka PhpMyAdmin), buat basis data dengan nama ``absensi-binbak''.
	\item Pastikan basis data ``absensi-binbak'' masih kosong, jika belum maka hapus seluruh tabel yang ada di dalamnya.
	\item Ubah \textit{username} dan \textit{password} untuk mengakses basis data MySql di ..\textbackslash xampp\textbackslash htdocs\textbackslash Absensi-Binbak\textbackslash app\textbackslash config\textbackslash database.php baris 59-60.
	\item Buka \textit{command prompt}.
	\item Ubah direktori \textit{command prompt} ke ..\textbackslash xampp\textbackslash htdocs\textbackslash Absensi-Binbak.
	\item Pastikan bahwa direktori ``public'' dapat dibaca dan ditulis.
	\item Ketik: \\\texttt{php artisan migrate}\\\texttt{php artisan db:seed}\\\texttt{php artisan serve}
	\item Buka sistem informasi tersebut di \textit{browser} (http://localhost:8000/).
\end{enumerate}



